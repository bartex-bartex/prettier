\documentclass{article}
\usepackage{graphicx} % Required for inserting images
\usepackage[T1]{fontenc} % Font encoding
\usepackage[absolute, overlay]{textpos}
\usepackage{caption}
\usepackage{float}
\usepackage{xcolor}
\usepackage[polish]{babel}
\usepackage[utf8]{inputenc}
\usepackage{hyperref}

\begin{document}
	\title{\fontsize{40}{48}\selectfont \textbf{\textcolor{blue!80!black}{Dokumentacja}}}
	\author{}
	\date{}

	\maketitle

	\begin{center}
		\textbf{\LARGE \textcolor{blue!80!black}{IoT Smart House}}
	\end{center}

	\vspace{2cm}

	\begin{center}
		\large \textbf{Autorzy:}

		\vspace{0.3cm}

		\begin{tabular}{c}
			Rafał Żelazko       \\
			Bartosz Sendor      \\
			Patryk Skowron      \\
			Bartosz Warchoł     \\
			Artur Wojciechowski \\
		\end{tabular}
	\end{center}

	\vspace{7cm}

	\begin{flushright}
		\large 04.2024
	\end{flushright}
	\newpage

	\maketitle

	\clearpage
	\tableofcontents
	\clearpage

	\section{Wprowadzenie}

	Witamy w świetle przyszłości domowej automatyzacji - \textbf{IoT Smart House}.
	Nasz wielomodułowy system będzie oferować kompleksowe rozwiązania zarządzania
	inteligentnym domem poprzez integrację wszystkich naszych urządzeń IoT różnych
	producentów w jednym miejscu przez wielu użytkowników jednocześnie.

	Zaawansowane \textbf{funkcje harmonogramowania} pracy urządzeń pozwolą na automatyczne
	uruchamianie ich w określonych godzinach i kolejności, a możliwość pisania
	własnych \textbf{skryptów} oraz instalowania \textbf{rozszerzeń} pozwoli
	użytkownikom na dostosowanie systemu dla ich indywidualnej wygody. Nasz system
	będzie również posiadać dodatkowe funkcje sterowania za pomocą \textbf{asystentów
	głosowych}, co znacznie ułatwi zarządzanie skomplikowanym systemem informacyjnym
	domu inteligentego.

	Celem IoT Smart House jest nie tylko wprowadzenie na rynek innowacyjnego rozwiązania,
	ale także czerpanie zysków z jego eksploatacji, co pozwoli zapewnić \textbf{wsparcie
	techniczne} oraz aktualizacje systemu \textbf{na długie lata}. Jednocześnie, aby
	sprostać zmieniającym się potrzebom użytkowników oraz nowym trendom technologicznym,
	nasz system będzie wspierać i promować kontrybycje twórców niezależnych
	poprzez wprowadzenie tzw. \textbf{sklepu społeczności}, gdzie programiści rozszerzeń
	oraz twórcy motywów będą mogli dzielić się swoją pracą.

	Bezpieczeństwo domu i rodziny to nasz największy priorytet. Dlatego też, nasza
	aplikacja będzie wspierać wiele funkcji bezpieczeństwa takich jak automatyczne
	powiadamianie, monitoring, czy alarm. Co się wiąże z bezpieczeństwem, wdrożymy
	zaawansowane \textbf{środki ochrony prywatności}, aby uniknąć udostępniania wrażliwych
	danych do większości systemów sterowanych przez IoT Smart House. Dodatkowo, w celu
	zwielokrotnienia skuteczności naszych zabezpieczeń, kod rdzenia naszego
	systemu będzie \textbf{w pełni zamknięty}, a jego artefakty będą
	rozpowszechniane tylko po ówczesnym zaciemnieniu.

	Dzięki swojej ponadczasowej jakości, nasza aplikacja będzie kontynuować swoją ewolucję
	przez dziesiątki lat, odpowiadając na zmieniające się potrzeby i oczekiwania
	użytkowników.

	\section{Przypadki użycia}
	\subsection{Aktorzy: }
	\begin{itemize}
		\item \textbf{Użytkownik bez konta} - użytkownik, który nie zarejestrował się
			jeszcze w aplikacji

		\item \textbf{Użytkownik zarejestrowany} - użytkownik, który nie zalogował się
			jeszcze w aplikacji, ale posiada konto

		\item \textbf{Użytkownik zalogowany} - użytkownik, który zalogował się do aplikacji

		\item \textbf{Członek wirtualnego domu} - użytkownik, który został przyjęty przez
			właścicielad do wirtualnego domu

		\item \textbf{Właściciel / współwłaściciel wirtualnego domu} - użytkownik, który
			stworzył wirtualny dom lub otrzymał uprawnienia od takiego użytkownika
	\end{itemize}
	\subsection{Przypadki użycia: }
	\begin{enumerate}
		\subsubsection{Logowanie i podstawowe opcje}
		\begin{enumerate}
			\item \textbf{Odebranie powiadomień} - odbieranie przez zalogowanego
				użytkownika powiadomień dotyczących urządzeń oraz stanu domu.
				\begin{table}[H]
					\centering
					\begin{tabular}{|c|p{7cm}|}
						\hline
						Nazwa                   & \textbf{Odebranie powiadomień}                                                                                                                                                                                                                                 \\
						\hline
						Opis                    & Odbieranie przez zalogowanego użytkownika powiadomień dotyczących urządzeń oraz stanu domu.                                                                                                                                                                    \\
						\hline
						Aktorzy                 & \begin{itemize}\item Członek wirtualnego domu\end{itemize}                                                                                                                                                                                                     \\
						\hline
						Warunki początkowe      & W domu doszło do zdarzenia które może wymagać ludzkiej interwencji.                                                                                                                                                                                            \\
						\hline
						Warunki końcowe sukcesu & Powiadomienie zostało dostarczone.                                                                                                                                                                                                                             \\
						\hline
						Warunki końcowe porażki & Brak. Powiadomienia są kolejkowane bez limitu.                                                                                                                                                                                                                 \\
						\hline
						Scenariusz główny       & \begin{enumerate}\item Użytkownik nie jest dostępny.

\item Następuje kolejkowanie powiadomień.

\item Użytkownik loguje się na swoje konto.

\item System kontynuuje czynności w scenariuszu głównym dla każdego skolejkowanego powiadomienia.\end{enumerate} \\
						\hline
						Scenariusz alternatywny & \begin{enumerate}\item Użytkownik nie jest dostępny.

\item Następuje kolejkowanie powiadomień.

\item Użytkownik loguje się na swoje konto.

\item System kontynuuje czynności w scenariuszu głównym dla każdego skolejkowanego powiadomienia.\end{enumerate} \\
						\hline
					\end{tabular}
				\end{table}

			\item \textbf{Zrealizowanie kodu dostępu dla gości} - realizacja przez
				użytkownika uprzednio wygenerowanego przez właściciela domu kodu
				umożliwiającego to, co właściciel wyznaczył.
				\begin{table}[H]
					\centering
					\begin{tabular}{|c|p{7cm}|}
						\hline
						Nazwa                   & \textbf{Zrealizowanie kodu dostępu dla gości}                                                                                                                                          \\
						\hline
						Opis                    & Realizacja przez użytkownika uprzednio wygenerowanego przez właściciela domu kodu umożliwiającego dostęp określony przez właściciela.                                                  \\
						\hline
						Aktorzy                 & \begin{itemize}\item Członek wirtualnego domu\end{itemize}                                                                                                                             \\
						\hline
						Warunki początkowe      & Właściciel domu wygenerował kod dostępu dla gościa.                                                                                                                                    \\
						\hline
						Warunki końcowe sukcesu & Gość uzyskał dostęp do domu.                                                                                                                                                           \\
						\hline
						Warunki końcowe porażki & Gość nie uzyskał dostępu do domu.                                                                                                                                                      \\
						\hline
						Scenariusz główny       & \begin{enumerate}\item Użytkownik otrzymuje od właściciela domu kod dostępu.

\item Użytkownik wprowadza kod dostępu w aplikacji.

\item System autoryzuje kod dostępu.\end{enumerate} \\
						\hline
						Scenariusz alternatywny & \begin{itemize}\item Kod dostępu jest niepoprawny, system odmawia dostępu.\end{itemize}                                                                                                \\
						\hline
					\end{tabular}
				\end{table}

			\item \textbf{Wylogowanie się z konta} - wylogowania się z konta przez
				użytkownika zalogowanego.
				\begin{table}[H]
					\centering
					\begin{tabular}{|c|p{7cm}|}
						\hline
						Nazwa                   & \textbf{Wylogowanie się z konta}                                                                                                                                                                         \\
						\hline
						Opis                    & Wylogowanie się z konta przez użytkownika zalogowanego.                                                                                                                                                  \\
						\hline
						Aktorzy                 & \begin{itemize}\item Członek wirtualnego domu\end{itemize}                                                                                                                                               \\
						\hline
						Warunki początkowe      & Użytkownik jest zalogowany.                                                                                                                                                                              \\
						\hline
						Warunki końcowe sukcesu & Użytkownik został wylogowany.                                                                                                                                                                            \\
						\hline
						Warunki końcowe porażki & Brak. Użytkownik zawsze zostanie wylogowany.                                                                                                                                                             \\
						\hline
						Scenariusz główny       & \begin{enumerate}\item Użytkownik jest zalogowany.

\item Użytkownik wybiera opcję wylogowania się.

\item System potwierdza wylogowanie i przekierowuje użytkownika na stronę logowania.\end{enumerate} \\
						\hline
						Scenariusz alternatywny & -                                                                                                                                                                                                        \\
						\hline
					\end{tabular}
				\end{table}

			\item \textbf{Reset swojego hasła} - reset hasła przez użytkownika
				zalogowanego.
				\begin{table}[H]
					\centering
					\begin{tabular}{|c|p{7cm}|}
						\hline
						Nazwa                   & Reset swojego hasła                                                                                                                                                                                                                                           \\
						\hline
						Opis                    & Resetowanie hasła przez użytkownika zalogowanego.                                                                                                                                                                                                             \\
						\hline
						Aktorzy                 & \begin{itemize}\item Członek wirtualnego domu\end{itemize}                                                                                                                                                                                                    \\
						\hline
						Warunki początkowe      & Użytkownik jest zalogowany.                                                                                                                                                                                                                                   \\
						\hline
						Warunki końcowe sukcesu & Hasło zostało zmienione.                                                                                                                                                                                                                                      \\
						\hline
						Warunki końcowe porażki & Hasło nie zostało zmienione.                                                                                                                                                                                                                                  \\
						\hline
						Scenariusz główny       & \begin{enumerate}\item Użytkownik jest zalogowany.

\item Użytkownik wybiera opcję resetowania hasła.

\item Użytkownik wprowadza nowe hasło i potwierdza.

\item System akceptuje nowe hasło i przekierowuje użytkownika na stronę logowania.\end{enumerate} \\
						\hline
						Scenariusz alternatywny & \begin{itemize}\item Użytkownik anuluje resetowanie hasła.\end{itemize}                                                                                                                                                                                       \\
						\hline
					\end{tabular}
				\end{table}

			\item \textbf{Edycja danych konta} - edycja danych konta przez użytkownika
				zalogowanego.
				\begin{table}[H]
					\centering
					\begin{tabular}{|c|p{7cm}|}
						\hline
						Nazwa                   & Edycja danych konta                                                                                                                                                                                                                 \\
						\hline
						Opis                    & Edycja danych konta przez użytkownika zalogowanego.                                                                                                                                                                                 \\
						\hline
						Aktorzy                 & \begin{itemize}\item Członek wirtualnego domu\end{itemize}                                                                                                                                                                          \\
						\hline
						Warunki początkowe      & Użytkownik jest zalogowany.                                                                                                                                                                                                         \\
						\hline
						Warunki końcowe sukcesu & Dane zostały zmienione.                                                                                                                                                                                                             \\
						\hline
						Warunki końcowe porażki & Dane nie zostały zmienione.                                                                                                                                                                                                         \\
						\hline
						Scenariusz główny       & \begin{enumerate}\item Użytkownik jest zalogowany.

\item Użytkownik wybiera opcję edycji danych konta.

\item Użytkownik wprowadza nowe dane i potwierdza.

\item System akceptuje nowe dane i zapisuje je w bazie.\end{enumerate} \\
						\hline
						Scenariusz alternatywny & \begin{itemize}\item Użytkownik anuluje edycję danych konta.

\item System odrzuca nowe dane jeśli są niepoprawne.\end{itemize}                                                                                                     \\
						\hline
					\end{tabular}
				\end{table}

			\item \textbf{Stworzenie wirtualnego domu} - utworzenie nowego wirtualnego
				domu przez użytkownika zalogowanego.
				\begin{table}[H]
					\centering
					\begin{tabular}{|c|p{7cm}|}
						\hline
						Nazwa                   & Stworzenie wirtualnego domu                                                                                                                                                                                                                                           \\
						\hline
						Opis                    & Utworzenie nowego wirtualnego domu przez użytkownika zalogowanego.                                                                                                                                                                                                    \\
						\hline
						Aktorzy                 & \begin{itemize}\item Właściciel wirtualnego domu\end{itemize}                                                                                                                                                                                                         \\
						\hline
						Warunki początkowe      & Użytkownik jest zalogowany.                                                                                                                                                                                                                                           \\
						\hline
						Warunki końcowe sukcesu & Wirtualny dom został utworzony.                                                                                                                                                                                                                                       \\
						\hline
						Warunki końcowe porażki & Wirtualny dom nie został utworzony.                                                                                                                                                                                                                                   \\
						\hline
						Scenariusz główny       & \begin{enumerate}\item Użytkownik jest zalogowany.

\item Użytkownik wybiera opcję stworzenia nowego wirtualnego domu.

\item Użytkownik wprowadza dane dotyczące nowego domu.

\item System tworzy nowy wirtualny dom i przypisuje go do użytkownika.\end{enumerate} \\
						\hline
						Scenariusz alternatywny & \begin{itemize}\item Użytkownik przerywa tworzenie nowego wirtualnego domu.\end{itemize}                                                                                                                                                                              \\
						\hline
					\end{tabular}
				\end{table}

			\item \textbf{Zainicjowanie przez użytkownika aktualizacji oprogramowania
				systemu} - zainicjowanie przez użytkownika aktualizacji oprogramowania
				przez użytkownika zalogowanego.
				\begin{table}[H]
					\centering
					\begin{tabular}{|c|p{7cm}|}
						\hline
						Nazwa                   & Zainicjowanie przez użytkownika aktualizacji oprogramowania systemu                                                                                                                                                                                                             \\
						\hline
						Opis                    & Zainicjowanie przez użytkownika aktualizacji oprogramowania systemu przez użytkownika zalogowanego.                                                                                                                                                                             \\
						\hline
						Aktorzy                 & \begin{itemize}\item Członek wirtualnego domu\end{itemize}                                                                                                                                                                                                                      \\
						\hline
						Warunki początkowe      & Użytkownik jest zalogowany.                                                                                                                                                                                                                                                     \\
						\hline
						Warunki końcowe sukcesu & Aktualizacja została zainstalowana.                                                                                                                                                                                                                                             \\
						\hline
						Warunki końcowe porażki & Aktualizacja nie została zainstalowana.                                                                                                                                                                                                                                         \\
						\hline
						Scenariusz główny       & \begin{enumerate}\item Użytkownik jest zalogowany.

\item Użytkownik wybiera opcję aktualizacji oprogramowania.

\item System sprawdza dostępność aktualizacji i rozpoczyna proces pobierania aktualizacji.

\item System instaluje pobraną paczkę aktualizacji.\end{enumerate} \\
						\hline
						Scenariusz alternatywny & \begin{itemize}\item Użytkownik anuluje pobieranie aktualizacji oprogramowania.\end{itemize}                                                                                                                                                                                    \\
						\hline
					\end{tabular}
				\end{table}

			\item \textbf{Zalogowanie się na konto} - zalogowanie się na konto przez
				użytkownika zarejestrowanego
				\begin{table}[H]
					\centering
					\begin{tabular}{|c|p{7cm}|}
						\hline
						Nazwa                   & Zalogowanie się na konto                                                                                                                                                                                                                                                                                \\
						\hline
						Opis                    & Zalogowanie się na konto przez użytkownika zarejestrowanego.                                                                                                                                                                                                                                            \\
						\hline
						Aktorzy                 & \begin{itemize}\item Użytkownik zarejestrowany\end{itemize}                                                                                                                                                                                                                                             \\
						\hline
						Warunki początkowe      & Użytkownik nie jest zalogowany.                                                                                                                                                                                                                                                                         \\
						\hline
						Warunki końcowe sukcesu & Użytkownik został zalogowany.                                                                                                                                                                                                                                                                           \\
						\hline
						Warunki końcowe porażki & Użytkownik nie został zalogowany.                                                                                                                                                                                                                                                                       \\
						\hline
						Scenariusz główny       & \begin{enumerate}\item Użytkownik otwiera stronę logowania.

\item Użytkownik wprowadza swoje dane logowania.

\item System autoryzuje użytkownika i przekierowuje go do panelu użytkownika.\end{enumerate}                                                                                             \\
						\hline
						Scenariusz alternatywny & \begin{enumerate}\item Dane logowania są nieprawidłowe, system wyświetla komunikat o błędzie.

\item System powiadamia adres e-mail powiązany z kontem o błędnej próbie logowania.

\item System powiadamia wszystkich użytkowników wirtualnych domów do których dodany jest użytkownik.\end{enumerate} \\
						\hline
					\end{tabular}
				\end{table}

			\item \textbf{Tworzenie konta} - tworzenie konta przez użytkownika bez
				konta.
				\begin{table}[H]
					\centering
					\begin{tabular}{|c|p{7cm}|}
						\hline
						Nazwa                   & Tworzenie konta                                                                                                                                                                                                                                                          \\
						\hline
						Opis                    & Tworzenie konta przez użytkownika bez konta.                                                                                                                                                                                                                             \\
						\hline
						Aktorzy                 & \begin{itemize}\item Użytkownik bez konta\end{itemize}                                                                                                                                                                                                                   \\
						\hline
						Warunki początkowe      & Użytkownik nie ma konta.                                                                                                                                                                                                                                                 \\
						\hline
						Warunki końcowe sukcesu & Konto zostało utworzone.                                                                                                                                                                                                                                                 \\
						\hline
						Warunki końcowe porażki & Konto nie zostało utworzone.                                                                                                                                                                                                                                             \\
						\hline
						Scenariusz główny       & \begin{enumerate}\item Użytkownik otwiera stronę rejestracji.

\item Użytkownik wprowadza dane rejestracyjne i potwierdza.

\item System tworzy nowe konto i przekierowuje na stronę logowania.\end{enumerate}                                                           \\
						\hline
						Scenariusz alternatywny & \begin{enumerate}\item Dane rejestracyjne nie spełniają wymagań lub już istnieje konto o podanych danych, system wyświetla komunikat o błędzie. Komunikat nigdy nie ujawnia tego, że inne konto już ma te same dane, tylko zawsze mówi o błędnych danych.\end{enumerate} \\
						\hline
					\end{tabular}
				\end{table}
		\end{enumerate}

		\subsubsection{Harmonogram, kontakty, powiadomienia}
		\begin{enumerate}
			\item \textbf{Dodanie / usunięcie wpisu z kalendarza domowego} - dodanie
				lub usunięcie wpisu z kalendarza domowego przez członka wirtualnego domu.

				\begin{table}[H]
					\centering
					\begin{tabular}{|c|p{7cm}|}
						\hline
						Nazwa                   & \textbf{Dodanie / usunięcie wpisu z kalendarza domowego}                                                                                                                                                                                                                                                                                                                                                                                             \\
						\hline
						Opis                    & dodanie lub usunięcie wpisu z kalendarza domowego przez członka wirtualnego domu                                                                                                                                                                                                                                                                                                                                                                     \\
						\hline
						Aktorzy                 & \begin{itemize}\item Członek wirtualnego domu\end{itemize}                                                                                                                                                                                                                                                                                                                                                                                           \\
						\hline
						Warunki początkowe      & Użytkownik ma prawo do edycji kalendarza                                                                                                                                                                                                                                                                                                                                                                                                             \\
						\hline
						Warunki końcowe sukcesu & Dodanie/usunięcie zostało wykonane                                                                                                                                                                                                                                                                                                                                                                                                                   \\
						\hline
						Warunki końcowe porażki & Brak zmiany w kalendarzu domowym                                                                                                                                                                                                                                                                                                                                                                                                                     \\
						\hline
						Scenariusz główny       & \begin{enumerate}\item Użytkownik, na głównym ekranie, wybiera opcję wyświetlenia kalendarza

\item Aplikacja wyświetla wszystkie wpisy dodane przez członków wirtualnego domu

\item Użytkownik może kliknąć "Dodaj nowy wpis" lub kliknąć jeden ze swoich wpisów

\item Aplikacja wyświetli formularz do edycji wpisu

\item Użytkownik może kliknąć opcję "Usuń wpis"

\item Aplikacja wyświetla powiadomienie o sukcesie operacji\end{enumerate} \\
						\hline
						Scenariusz alternatywny & \begin{itemize}\item AD. 5: Gdy użytkownik spróbuje usunąć nieswój wpis, to aplikacja o tym poinformuje i anuluje operację.\end{itemize}                                                                                                                                                                                                                                                                                                             \\
						\hline
					\end{tabular}
				\end{table}

			\item \textbf{Usunięcie wpisów innych domowników kalendarza} - usunięcie
				przez właściciela wpisów innych członków w celu zwiększenia czytelności
				kalendarza

				\begin{table}[H]
					\centering
					\begin{tabular}{|c|p{7cm}|}
						\hline
						Nazwa                   & \textbf{Usunięcie wpisów innych domowników kalendarza}                                                                                                                                                                                                                                                                                 \\
						\hline
						Opis                    & usunięcie przez właściciela wpisów innych członków w celu zwiększenia czytelności kalendarza                                                                                                                                                                                                                                           \\
						\hline
						Aktorzy                 & \begin{itemize}\item Właściciel wirtualnego domu\end{itemize}                                                                                                                                                                                                                                                                          \\
						\hline
						Warunki początkowe      & W kalendarzu istnieją wpisy członków wirtualnego domu                                                                                                                                                                                                                                                                                  \\
						\hline
						Warunki końcowe sukcesu & Wpis został usunięty                                                                                                                                                                                                                                                                                                                   \\
						\hline
						Warunki końcowe porażki & Wpisy kalendarza nie uległy zmianie                                                                                                                                                                                                                                                                                                    \\
						\hline
						Scenariusz główny       & \begin{enumerate}\item Użytkownik, na głównym ekranie, wybiera opcję wyświetlenia kalendarza

\item Aplikacja wyświetla wszystkie wpisy dodane przez członków wirtualnego domu

\item Jeżeli użytkownik jest właścicielem domu to może usunąc dowolny wpis

\item Aplikacja wyświetla powiadomienie o sukcesie operacji\end{enumerate} \\
						\hline
						Scenariusz alternatywny & \begin{itemize}\item AD. 3: Jeżeli użytkownik nie jest właścicielem, to pojawi się powiadomienie, a operacja anulowana.\end{itemize}                                                                                                                                                                                                   \\
						\hline
					\end{tabular}
				\end{table}

			\item \textbf{Stworzenie niestandardowego harmonogramu} - stworzenie
				niestandardowego harmonogramu pracy urządzeń IoT.

				\begin{table}[H]
					\centering
					\begin{tabular}{|c|p{7cm}|}
						\hline
						Nazwa                   & \textbf{Stworzenie niestandardowego harmonogramu}                                                                                                                                                                                                                                                                                                    \\
						\hline
						Opis                    & stworzenie niestandardowego harmonogramu pracy urządzeń IoT                                                                                                                                                                                                                                                                                          \\
						\hline
						Aktorzy                 & \begin{itemize}\item Członek wirtualnego domu\end{itemize}                                                                                                                                                                                                                                                                                           \\
						\hline
						Warunki początkowe      & -                                                                                                                                                                                                                                                                                                                                                    \\
						\hline
						Warunki końcowe sukcesu & Dodano harmonogram                                                                                                                                                                                                                                                                                                                                   \\
						\hline
						Warunki końcowe porażki & Harmonogram nie został dodany                                                                                                                                                                                                                                                                                                                        \\
						\hline
						Scenariusz główny       & \begin{enumerate}\item Użytkownik, na głównym ekranie, wybiera opcję wyświetlenia listy harmonogramów pracy urządzeń IoT

\item Użytkownik może kliknąć opcję "Dodaj nowy harmonogram"

\item Aplikacja wyświetla edytor harmonogramów

\item Użytkownik używając narzędzia tworzy harmonogram

\item Użytkownik zapisuje harmonogram\end{enumerate} \\
						\hline
						Scenariusz alternatywny & \begin{itemize}\item AD. 1: Jeżeli lista jest pusta, aplikacja wyświetli taką informację\end{itemize}                                                                                                                                                                                                                                                \\
						\hline
					\end{tabular}
				\end{table}

			\item \textbf{Edycja niestandardowego harmonogramu} - edycja
				niestandardowego harmonogramu pracy urządzeń IoT.

				\begin{table}[H]
					\centering
					\begin{tabular}{|c|p{7cm}|}
						\hline
						Nazwa                   & Edycja niestandardowego harmonogramu                                                                                                                                                                                                                                                                                                                              \\
						\hline
						Opis                    & edycja niestandardowego harmonogramu pracy urządzeń IoT                                                                                                                                                                                                                                                                                                           \\
						\hline
						Aktorzy                 & \begin{itemize}\item Właściciel wirtualnego domu

\item Członek wirtualnego domu\end{itemize}                                                                                                                                                                                                                                                                     \\
						\hline
						Warunki początkowe      & Jeżeli członek wirtualnego domu, to musi posiadać swoje harmonogramy                                                                                                                                                                                                                                                                                              \\
						\hline
						Warunki końcowe sukcesu & Zedytowano harmonogram                                                                                                                                                                                                                                                                                                                                            \\
						\hline
						Warunki końcowe porażki & Brak zmiany w harmonogramie                                                                                                                                                                                                                                                                                                                                       \\
						\hline
						Scenariusz główny       & \begin{enumerate}\item Użytkownik, na głównym ekranie, wybiera opcję wyświetlenia listy harmonogramów pracy urządzeń IoT

\item Użytkownik może kliknąć dany harmonogram w celu jego edycji

\item Aplikacja wyświetla edytor harmonogramów

\item Użytkownik używając narzędzia nanosi poprawki do harmonogramu

\item Użytkownik zapisuje zmiany\end{enumerate} \\
						\hline
						Scenariusz alternatywny & \begin{itemize}\item AD. 1: Jeżeli lista jest pusta, aplikacja wyświetli taką informację

\item AD. 2: Jeżeli użytkownik nie jest właścicielem domu, to edytować może wyłącznie swoje harmonogramy\end{itemize}                                                                                                                                                   \\
						\hline
					\end{tabular}
				\end{table}

			\item \textbf{Zmiana statusu harmonogramu pracy} - zmieniając status można
				m.in. aktywować lub dezaktywować harmonogram

				\begin{table}[H]
					\centering
					\begin{tabular}{|c|p{7cm}|}
						\hline
						Nazwa                   & Zmiana statusu harmonogramu pracy                                                                                                                                                                                                                                                                                                                   \\
						\hline
						Opis                    & zmieniając status można m.in. aktywować lub dezaktywować harmonogram                                                                                                                                                                                                                                                                                \\
						\hline
						Aktorzy                 & \begin{itemize}\item Członek wirtualnego domu\end{itemize}                                                                                                                                                                                                                                                                                          \\
						\hline
						Warunki początkowe      & Harmonogram nie został wyłączony przez właściciela wirtualnego domu                                                                                                                                                                                                                                                                                 \\
						\hline
						Warunki końcowe sukcesu & Harmonogram zmienił status                                                                                                                                                                                                                                                                                                                          \\
						\hline
						Warunki końcowe porażki & Harmonogram nie zmienił statusu                                                                                                                                                                                                                                                                                                                     \\
						\hline
						Scenariusz główny       & \begin{enumerate}\item Użytkownik, na głównym ekranie, wybiera opcję wyświetlenia listy harmonogramów pracy urządzeń IoT

\item Użytkownik może kliknąć w ikonkę statusu wybranego harmonogramu

\item Aplikacja wyświetla możliwe statusy do ustawienia

\item Użytkownik wybiera opcję "Aktywuj"

\item Użytkownik zapisuje zmiany\end{enumerate} \\
						\hline
						Scenariusz alternatywny & \begin{itemize}\item AD. 1: Jeżeli lista jest pusta, aplikacja wyświetli taką informację

\item AD. 4: Jeżeli harmonogram wyłączył właściciel, to inni użytkownicy nie będą mieli opcji "Aktywuj"\end{itemize}                                                                                                                                      \\
						\hline
					\end{tabular}
				\end{table}

			\item \textbf{Konfiguracja listy kontaktów do powiadamiania w nagłych
				sytuacjach} - konfiguracja listy kontaktów do powiadamiania w nagłych
				sytuacjach przez członka wirtualnego domu.

				\begin{table}[H]
					\centering
					\begin{tabular}{|c|p{7cm}|}
						\hline
						Nazwa                   & Konfiguracja listy kontaktów do powiadamiania w nagłych sytuacjach                                                                                                                                                                                                                                                                                                                                                                                                                                                                                       \\
						\hline
						Opis                    & konfiguracja listy kontaktów do powiadamiania w nagłych sytuacjach przez członka wirtualnego domu                                                                                                                                                                                                                                                                                                                                                                                                                                                        \\
						\hline
						Aktorzy                 & \begin{itemize}\item Członek wirtualnego domu\end{itemize}                                                                                                                                                                                                                                                                                                                                                                                                                                                                                               \\
						\hline
						Warunki początkowe      & -                                                                                                                                                                                                                                                                                                                                                                                                                                                                                                                                                        \\
						\hline
						Warunki końcowe sukcesu & Zmiana w liście kontaktów                                                                                                                                                                                                                                                                                                                                                                                                                                                                                                                                \\
						\hline
						Warunki końcowe porażki & Brak zmian w liście kontaków                                                                                                                                                                                                                                                                                                                                                                                                                                                                                                                             \\
						\hline
						Scenariusz główny       & \begin{enumerate}\item Użytkownik, na głównym ekranie, wybiera opcję "Ustawienia"

\item Aplikacja wyświetla listę Ustawień

\item Użytkownik może kliknąć w opcję "Powiadomienia w razie nagłych sytuacji"

\item Aplikacja wyświetla listę ustawień

\item Użytkownik wybiera opcję "Konfiguruj listę kontaktów"

\item Aplikacja wyświetla listę kontaktów

\item Użytkownik może wybrać dodanie lub edycję kontaktu

\item Aplikacji wyświetlarz formularz z danymi

\item Użytkownik zapisuje dane

\item Aplikacje uaktualnia listę\end{enumerate} \\
						\hline
						Scenariusz alternatywny & -                                                                                                                                                                                                                                                                                                                                                                                                                                                                                                                                                        \\
						\hline
					\end{tabular}
				\end{table}

			\item \textbf{Wyciszanie powiadomień na określony czas} - wyciszanie
				powiadomień na określony czas przez członka wirtualnego domu.
		\end{enumerate}

		\begin{table}[H]
			\centering
			\begin{tabular}{|c|p{7cm}|}
				\hline
				Nazwa                   & Wyciszanie powiadomień na określony czas                                                                                                                                                                                                                                                                                                                                                                                                                                 \\
				\hline
				Opis                    & wyciszanie powiadomień na określony czas przez członka wirtualnego domu                                                                                                                                                                                                                                                                                                                                                                                                  \\
				\hline
				Aktorzy                 & \begin{itemize}\item Członek wirtualnego domu\end{itemize}                                                                                                                                                                                                                                                                                                                                                                                                               \\
				\hline
				Warunki początkowe      & -                                                                                                                                                                                                                                                                                                                                                                                                                                                                        \\
				\hline
				Warunki końcowe sukcesu & Wyciszono powiadomienia na określony czas                                                                                                                                                                                                                                                                                                                                                                                                                                \\
				\hline
				Warunki końcowe porażki & Powiadomienia nie zostały wyciszone                                                                                                                                                                                                                                                                                                                                                                                                                                      \\
				\hline
				Scenariusz główny       & \begin{enumerate}\item Użytkownik, na głównym ekranie, wybiera opcję "Ustawienia"

\item Aplikacja wyświetla listę Ustawień

\item Użytkownik może kliknąć w opcję "Powiadomienia"

\item Aplikacja wyświetla listę ustawień

\item Użytkownik może kliknąc w opcję "Wycisz"

\item Aplikacja wyświetla okno z zapytaniem na jaki okres

\item Użytkownik może wpisać okres czasu i zaakceptować

\item Aplikacja Wycisza powiadomienia na określony czas\end{enumerate} \\
				\hline
				Scenariusz alternatywny & \begin{itemize}\item AD. 7: W przypadku błędnego formatu aplikacja poprosi o ponownie wpisanie okresu\end{itemize}                                                                                                                                                                                                                                                                                                                                                       \\
				\hline
			\end{tabular}
		\end{table}

		\subsubsection{Wykorzystanie urzadzeń IoT}
		\begin{enumerate}
			\item \textbf{Odblokowanie inteligentnego zamka do drzwi z aplikacji} -
				odblokowanie inteligentnego zamka do drzwi z poziomu aplikacji przez
				członka wirtualnego domu.
				\begin{table}[H]
					\centering
					\begin{tabular}{|c|p{7cm}|}
						\hline
						Nazwa                   & \textbf{Odblokowanie inteligentnego zamka do drzwi z aplikacji}                                                                                                                                          \\
						\hline
						Opis                    & odblokowanie inteligentnego zamka do drzwi z poziomu aplikacji przez członka wirtualnego domu                                                                                                            \\
						\hline
						Aktorzy                 & \begin{itemize}\item Członek wirtualnego domu\end{itemize}                                                                                                                                               \\
						\hline
						Warunki początkowe      & Zamek drzwi frontowych jest zablokowany                                                                                                                                                                  \\
						\hline
						Warunki końcowe sukcesu & Zamek drzwi frontowych został odblokowany                                                                                                                                                                \\
						\hline
						Warunki końcowe porażki & Zamek drzwi frontowych jest zablokowany                                                                                                                                                                  \\
						\hline
						Scenariusz główny       & \begin{enumerate}\item Klient wysyła zapytanie do serwera o odblokowanie zamka

\item Serwer sprawdza permisje stałe oraz nadane na czas określony

\item Serwer otwiera zamek w drzwiach\end{enumerate} \\
						\hline
						Scenariusz alternatywny & \begin{itemize}\item Ad ii. Serwer stwierdza, że klient nie posiada uprawnień i odmawia otwarcia. Koniec\end{itemize}                                                                                    \\
						\hline
					\end{tabular}
				\end{table}

			\item \textbf{Zarządzanie roletami w domu} - zarządzanie systemem rolet
				przez członka wirtualnego domu.
				\begin{table}[H]
					\centering
					\begin{tabular}{|c|p{7cm}|}
						\hline
						Nazwa                   & \textbf{Zarządzanie roletami w domu}                                                                                                                                                                                                                          \\
						\hline
						Opis                    & zarządzanie systemem rolet przez członka wirtualnego domu                                                                                                                                                                                                     \\
						\hline
						Aktorzy                 & \begin{itemize}\item Członek wirtualnego domu\end{itemize}                                                                                                                                                                                                    \\
						\hline
						Warunki początkowe      & Rolety mogą być użytkowane (np. są połączone z systemem)                                                                                                                                                                                                      \\
						\hline
						Warunki końcowe sukcesu & Rolety zostały dostosowane do nowych wytycznych                                                                                                                                                                                                               \\
						\hline
						Warunki końcowe porażki & Stan rolet się nie zmienił lub zmienił się, ale nie spełnił wszystkich wymagań klienta                                                                                                                                                                        \\
						\hline
						Scenariusz główny       & \begin{enumerate}\item Klient wysyła zapytanie do serwera o wyświetlenie aktualnego stanu rolet

\item Serwer wysyła aktualny stan rolet

\item Klient zgłasza modyfikacje jakie chce wprowadzić w stanie rolet

\item Serwer wprowadza zmiany\end{enumerate} \\
						\hline
						Scenariusz alternatywny & \begin{itemize}\item Ad (iv) Serwer zwraca informację zwrotną, że roleta nie może być zamknięta dopóki okno jest otwarte. Koniec\end{itemize}                                                                                                                 \\
						\hline
					\end{tabular}
				\end{table}

			\item \textbf{Podgląd obrazu z kamery monitoringu przy drzwiach
				wejściowych} - podgląd kamery frontowej przez członka wirtualnego domu.
				\begin{table}[H]
					\centering
					\begin{tabular}{|c|p{7cm}|}
						\hline
						Nazwa                   & \textbf{Podgląd obrazu z kamery monitoringu przy drzwiach wejściowych}                                                                                                                                                                              \\
						\hline
						Opis                    & podgląd kamery frontowej przez członka wirtualnego domu                                                                                                                                                                                             \\
						\hline
						Aktorzy                 & \begin{itemize}\item Członek wirtualnego domu\end{itemize}                                                                                                                                                                                          \\
						\hline
						Warunki początkowe      & Kamera jest gotowa do użytkowania (jest podłączona i nie jest użytkowana przez innych członków domu)                                                                                                                                                \\
						\hline
						Warunki końcowe sukcesu & Obraz z kamery został przekazany klientowi                                                                                                                                                                                                          \\
						\hline
						Warunki końcowe porażki & Obraz z kamery nie został przekazany klientowi                                                                                                                                                                                                      \\
						\hline
						Scenariusz główny       & \begin{enumerate}\item Klient wysyła zapytanie do serwera o wyświetlenie widoku z kamery

\item Serwer rozpoczyna streaming z kamery drzwi frontowych

\item Klient zgłasza zakończenie oglądania

\item Serwer zatrzymuje streaming\end{enumerate} \\
						\hline
						Scenariusz alternatywny & \begin{itemize}\item Ad (ii) Inny klient użytkuje obecnie kamerę. Koniec

\item Ad (iii) Klient nie odpowiada, serwer zatrzymuje streaming. Koniec\end{itemize}                                                                                     \\
						\hline
					\end{tabular}
				\end{table}

				\begin{figure}[h]
					\centering
					\includegraphics[width=0.36\paperwidth]{diag_czynn_obraz_z_kamer.png}
				\end{figure}

			\item \textbf{Przeprowadzanie połączeń głosowych z kamerą przy drzwiach} -
				prowadzenie wideo rozmów w aplikacji z osobą stojącą przed drzwiami
				frontowymi przez członka wirtualnego domu.
				\begin{table}[H]
					\centering
					\begin{tabular}{|c|p{7cm}|}
						\hline
						Nazwa                   & \textbf{Przeprowadzanie połączeń głosowych z kamerą przy drzwiach}                                                                                                                                                                                                       \\
						\hline
						Opis                    & prowadzenie wideo rozmów w aplikacji z osobą stojącą przed drzwiami frontowymi przez członka wirtualnego domu                                                                                                                                                            \\
						\hline
						Aktorzy                 & \begin{itemize}\item Członek wirtualnego domu

\item Użytkownik niezalogowany\end{itemize}                                                                                                                                                                               \\
						\hline
						Warunki początkowe      & Kamera jest gotowa do użytkowania (jest podłączona)                                                                                                                                                                                                                      \\
						\hline
						Warunki końcowe sukcesu & Nawiązano połączenie między kamerą - użytkownik niezalogowany oraz członkiem domu                                                                                                                                                                                        \\
						\hline
						Warunki końcowe porażki & Nie nawiązano połączenia między kamerą - użytkownik niezalogowany oraz członkiem domu                                                                                                                                                                                    \\
						\hline
						Scenariusz główny       & \begin{enumerate}\item Urządzenie telekomunikacyjne drzwi wysyła zapytanie o połączenie video z członkiem domu

\item Serwer rozpoczyna połączenie kamery drzwi frontowych z członkiem

\item Członek odbiera połączenie

\item Członek kończy połączenie\end{enumerate} \\
						\hline
						Scenariusz alternatywny & \begin{itemize}\item Ad (iii) Żaden członek nie odebrał połączenia. Koniec

\item Ad (iv) Utrata połączenia z Członkiem. Koniec\end{itemize}                                                                                                                             \\
						\hline
					\end{tabular}
				\end{table}

			\item \textbf{Odblokowanie dostępu do smart TV dla dzieci w określonych
				godzinach} - zarządzanie dostępem do telewizora dla dzieci przez członka
				wirtualnego domu z odpowiednimi uprawnieniami.
				\begin{table}[H]
					\centering
					\begin{tabular}{|c|p{7cm}|}
						\hline
						Nazwa                   & \textbf{Odblokowanie dostępu do smart TV dla dzieci w określonych godzinach}                                                                                                                            \\
						\hline
						Opis                    & zarządzanie dostępem do telewizora dla dzieci przez członka wirtualnego domu z odpowiednimi uprawnieniami                                                                                               \\
						\hline
						Aktorzy                 & \begin{itemize}\item Członek wirtualnego domu

\item Członek wirtualnego domu o roli dziecka\end{itemize}                                                                                               \\
						\hline
						Warunki początkowe      & W domu są użytkownicy o roli dziecka                                                                                                                                                                    \\
						\hline
						Warunki końcowe sukcesu & Zmieniono dostęp do smart TV według wytycznych klienta                                                                                                                                                  \\
						\hline
						Warunki końcowe porażki & Nie zmieniono dostępu do smart TV według wytycznych klienta                                                                                                                                             \\
						\hline
						Scenariusz główny       & \begin{enumerate}\item Klient wysyła zapytanie o odblokowanie dostępu do smart TV dla dzieci w określonych godzinach

\item Serwer uaktualnia harmonogram dostępu do smart TV dla dzieci\end{enumerate} \\
						\hline
						Scenariusz alternatywny & -                                                                                                                                                                                                       \\
						\hline
					\end{tabular}
				\end{table}

			\item \textbf{Zmiana ustawień termostatu i klimatyzacji w pomieszczeniach}
				- zarządzanie ustawieniami termostatu i klimatyzacji przez członka
				wirtualnego domu.
				\begin{table}[H]
					\centering
					\begin{tabular}{|c|p{7cm}|}
						\hline
						Nazwa                   & \textbf{Zmiana ustawień termostatu i klimatyzacji w pomieszczeniach}                                                                                                                                                                                                                                                                                                                         \\
						\hline
						Opis                    & zarządzanie ustawieniami termostatu i klimatyzacji przez członka wirtualnego domu                                                                                                                                                                                                                                                                                                            \\
						\hline
						\hline
						Aktorzy                 & \begin{itemize}\item Członek wirtualnego domu\end{itemize}                                                                                                                                                                                                                                                                                                                                   \\
						\hline
						Warunki początkowe      & Termostat i klimatyzacja mogą być użytkowane (np. są połączone z systemem)                                                                                                                                                                                                                                                                                                                   \\
						\hline
						Warunki końcowe sukcesu & Termostat i klimatyzacja zostały dostosowane do nowych wytycznych                                                                                                                                                                                                                                                                                                                            \\
						\hline
						Warunki końcowe porażki & Stan termostatu i klimatyzacji się nie zmienił lub zmienił się, ale nie spełnił wszystkich wymagań klienta                                                                                                                                                                                                                                                                                   \\
						Scenariusz główny       & \begin{enumerate}\item Klient wysyła zapytanie do serwera o wyświetlenie aktualnego stanu termostatu i klimatyzacji w pomieszczeniach

\item Serwer wysyła aktualny stan termostatu, klimatyzacji i temperatury w pomieszczeniach

\item Klient zgłasza modyfikacje jakie chce wprowadzić w stanie termostatu i klimatyzacji w pomieszczeniach

\item Serwer wprowadza zmiany\end{enumerate} \\
						\hline
						Scenariusz alternatywny & \begin{itemize}\item Ad (iv) Serwer wysyła informację zwrotną, że nie można włączyć klimatyzacji, ponieważ w pomieszczeniu są otwarte okna. Koniec\end{itemize}                                                                                                                                                                                                                              \\
						\hline
					\end{tabular}
				\end{table}

			\item \textbf{Sprawdzanie stanu samochodów w wirtualnym domu} -
				sprawdzanie stanu samochodów przez członka wirtualnego domu.
				\begin{table}[H]
					\centering
					\begin{tabular}{|c|p{7cm}|}
						\hline
						Nazwa                   & \textbf{Sprawdzanie stanu samochodów w wirtualnym domu}                                                                                                                                                                                          \\
						\hline
						Opis                    & sprawdzanie stanu samochodów przez członka wirtualnego domu                                                                                                                                                                                      \\
						\hline
						Aktorzy                 & \begin{itemize}\item Członek wirtualnego domu\end{itemize}                                                                                                                                                                                       \\
						\hline
						Warunki początkowe      & Samochody mogą udostępniać dane (np. są połączone z systemem)                                                                                                                                                                                    \\
						\hline
						Warunki końcowe sukcesu & Dane od samochodów zostały zebrane i przekazane klientowi                                                                                                                                                                                        \\
						\hline
						Warunki końcowe porażki & Dane od samochodów nie zostały zebrane lub nie zostały przekazane klientowi                                                                                                                                                                      \\
						\hline
						Scenariusz główny       & \begin{enumerate}\item Klient wysyła zapytanie do serwera o wyświetlenie aktualnego stanu samochodów

\item Serwer wysyła zapytanie do samochodów o podanie swoich aktualnych stanów

\item Serwer wysyła zebrane dane do klienta\end{enumerate} \\
						\hline
						Scenariusz alternatywny & \begin{itemize}\item Ad (ii) Serwer nie znajduje adekwatnych pojazdów i wysyła odpowiednią informację zwrotną. Koniec\end{itemize}                                                                                                               \\
						\hline
					\end{tabular}
				\end{table}

				\begin{figure}[h]
					\centering
					\includegraphics[width=0.36\paperwidth]{
						diag_czynn_stany_samochodow.png
					}
				\end{figure}

			\item \textbf{Możliwość uruchomienia świateł} - możliwość uruchomienia
				świateł z poziomu aplikacji przez członka wirtualnego domu.
				\begin{table}[H]
					\centering
					\begin{tabular}{|c|p{7cm}|}
						\hline
						Nazwa                   & \textbf{Możliwość uruchomienia świateł}                                                                                                                                                                                                                             \\
						\hline
						Opis                    & możliwość uruchomienia świateł z poziomu aplikacji przez członka wirtualnego domu                                                                                                                                                                                   \\
						\hline
						Aktorzy                 & \begin{itemize}\item Członek wirtualnego domu\end{itemize}                                                                                                                                                                                                          \\
						\hline
						Warunki początkowe      & Światła mogą być użytkowane (np. są połączone z systemem)                                                                                                                                                                                                           \\
						\hline
						Warunki końcowe sukcesu & Światła zostały dostosowane do nowych wytycznych                                                                                                                                                                                                                    \\
						\hline
						Warunki końcowe porażki & Stan świateł się nie zmienił lub zmienił się, ale nie spełnił wszystkich wymagań klienta                                                                                                                                                                            \\
						\hline
						Scenariusz główny       & \begin{enumerate}\item Klient wysyła zapytanie do serwera o wyświetlenie aktualnego stanu świateł

\item Serwer wysyła aktualny stan świateł

\item Klient zgłasza modyfikacje jakie chce wprowadzić w stanie świateł

\item Serwer wprowadza zmiany\end{enumerate} \\
						\hline
						Scenariusz alternatywny & \begin{itemize}\item Ad (iv) Serwer nie może wprowadzić zmian i wysyła odpowiednią informację zwrotną do klienta. Koniec\end{itemize}                                                                                                                               \\
						\\
						\hline
					\end{tabular}
				\end{table}

			\item \textbf{Zarządzanie AGD} - zarządzanie AGD z poziomu aplikacji przez
				członka wirtualnego domu.
				\begin{table}[H]
					\centering
					\begin{tabular}{|c|p{7cm}|}
						\hline
						Nazwa                   & \textbf{Zarządzanie AGD}                                                                                                                                                                                                                                                           \\
						\hline
						Opis                    & możliwość uruchomienia urządzeń AGD z poziomu aplikacji przez członka wirtualnego domu                                                                                                                                                                                             \\
						\hline
						Aktorzy                 & \begin{itemize}\item Członek wirtualnego domu\end{itemize}                                                                                                                                                                                                                         \\
						\hline
						Warunki początkowe      & Urządzenia AGD mogą być użytkowane (np. są połączone z systemem)                                                                                                                                                                                                                   \\
						\hline
						Warunki końcowe sukcesu & Urządzenia AGD zostały dostowowane do nowych wytycznych                                                                                                                                                                                                                            \\
						\hline
						Warunki końcowe porażki & Stan urządzeń AGD się nie zmienił lub zmienił się, ale nie spełnił wszystkich wymagań klienta                                                                                                                                                                                      \\
						\hline
						Scenariusz główny       & \begin{enumerate}\item Klient wysyła zapytanie do serwera o wyświetlenie aktualnego stanu urządzeń AGD

\item Serwer wysyła aktualny stan urządzeń AGD

\item Klient zgłasza modyfikacje jakie chce wprowadzić w stanie urządzeń AGD

\item Serwer wprowadza zmiany\end{enumerate} \\
						\hline
						Scenariusz alternatywny & \begin{itemize}\item Ad (iv) Serwer nie może wprowadzić zmian i wysyła odpowiednią informację zwrotną do klienta. Koniec\end{itemize}                                                                                                                                              \\
						\hline
					\end{tabular}
				\end{table}

			\item \textbf{Zarządzanie monitoringiem} - zarządzanie monitoringiem przez
				członka wirtualnego domu z odpowiednimi uprawnieniami.
				\begin{table}[H]
					\centering
					\begin{tabular}{|c|p{7cm}|}
						\hline
						Nazwa                   & \textbf{Zarządzanie monitoringiem}                                                                                                                                                                                                                                              \\
						\hline
						Opis                    & możliwość zarządzania monitoringiem z poziomu aplikacji przez członka wirtualnego domu                                                                                                                                                                                          \\
						\hline
						Aktorzy                 & \begin{itemize}\item Członek wirtualnego domu\end{itemize}                                                                                                                                                                                                                      \\
						\hline
						Warunki początkowe      & Urządzenia monitoringu mogą być użytkowane (np. są połączone z systemem)                                                                                                                                                                                                        \\
						\hline
						Warunki końcowe sukcesu & Urządzenia monitoringu zostały dostowowane do nowych wytycznych                                                                                                                                                                                                                 \\
						\hline
						Warunki końcowe porażki & Stan urządzeń monitoringu się nie zmienił lub zmienił się, ale nie spełnił wszystkich wymagań klienta                                                                                                                                                                           \\
						\hline
						Scenariusz główny       & \begin{enumerate}\item Klient wysyła zapytanie do serwera o wyświetlenie aktualnego stanu monitoringu

\item Serwer wysyła aktualny stan monitoringu

\item Klient zgłasza modyfikacje jakie chce wprowadzić w stanie monitoringu

\item Serwer wprowadza zmiany\end{enumerate} \\
						\hline
						Scenariusz alternatywny & \begin{itemize}\item Ad (iv) Serwer nie może wprowadzić zmian i wysyła odpowiednią informację zwrotną do klienta. Koniec\end{itemize}                                                                                                                                           \\
						\hline
					\end{tabular}
				\end{table}

				\begin{figure}[h]
					\centering
					\includegraphics[width=0.36\paperwidth]{giad_czynn_zarz_monitor.png}
				\end{figure}

			\item \textbf{Pobranie danych z monitoringu} przez członka wirtualnego
				domu z odpowiednimi uprawnieniami.
				\begin{table}[H]
					\centering
					\begin{tabular}{|c|p{7cm}|}
						\hline
						Nazwa                   & \textbf{Pobranie danych z monitoringu}                                                                                                                                   \\
						\hline
						Opis                    & możliwość pobrania danych z monitoringu z poziomu aplikacji                                                                                                              \\
						\hline
						Aktorzy                 & \begin{itemize}\item Członek wirtualnego domu\end{itemize}                                                                                                               \\
						\hline
						Warunki początkowe      & Urządzenia przechowujące nagrania monitoringu mogą być użytkowane (np. są połączone z systemem)                                                                          \\
						\hline
						Warunki końcowe sukcesu & Nagrania monitoringu zostały przekazane klientowi                                                                                                                        \\
						\hline
						Warunki końcowe porażki & Nagrania monitoringu nie zostały przekazane klientowi                                                                                                                    \\
						\hline
						Scenariusz główny       & \begin{enumerate}\item Klient wysyła zapytanie do serwera o pobranie danych z konkretnego okresu czasu z monitoringu

\item Serwer wysyła żadane nagrania\end{enumerate} \\
						\hline
						Scenariusz alternatywny & \begin{itemize}\item Ad (ii) Serwer nie posiada żądanych nagrań, zwraca odpowiednią informację. Koniec\end{itemize}                                                      \\
						\hline
					\end{tabular}
				\end{table}
		\end{enumerate}

		\subsubsection{Automatyzacja i statystyki}
		\begin{enumerate}
			\item \textbf{Sprawdzanie statystyk wirtualnego domu} - możliwość
				sprawdzenia statystyk wirtualnego domu przez członka wirtualnego domu.

				\begin{table}[H]
					\centering
					\begin{tabular}{|c|p{7cm}|}
						\hline
						Nazwa                   & \textbf{Sprawdzanie statystyk wirtualnego domu}                                                                                                                                                                                                                                               \\
						\hline
						Opis                    & możliwość sprawdzenia statystyk wirtualnego domu przez członka wirtualnego domu                                                                                                                                                                                                               \\
						\hline
						Aktorzy                 & \begin{itemize}\item Właściciel/współwłaściciel domu\end{itemize}                                                                                                                                                                                                                             \\
						\hline
						Warunki początkowe      & Użytkownicy korzystali wcześniej z usług wirtualnego domu przez jakiś czas                                                                                                                                                                                                                    \\
						\hline
						Warunki końcowe sukcesu & Użytkownikowi zostają przedstawione statystyki wirtualnego domu                                                                                                                                                                                                                               \\
						\hline
						Warunki końcowe porażki & Użytkownik nie otrzymał dostępu do statystyk.                                                                                                                                                                                                                                                 \\
						\hline
						Scenariusz główny       & \begin{enumerate}\item Użytkownik wybiera opcje "statystyki"

\item Użytkownik wysyła zapytanie do serwera o wyświetlenie zużytych zasobów jego domu

\item Aplikacja wyświetla zużyte przez członków domu zasoby

\item Użytkownik zamyka statystyki po zakończeniu oglądania\end{enumerate} \\
						\hline
						Scenariusz alternatywny & -                                                                                                                                                                                                                                                                                             \\
						\hline
					\end{tabular}
				\end{table}

			\item \textbf{Wyświatlenie użytkowników znajdujących się w domu} -
				uzyskanie listy użytkowników obecnych w domu przez członka o odpowiednio
				nadanych uprawnieniach

				\begin{table}[H]
					\centering
					\begin{tabular}{|c|p{7cm}|}
						\hline
						Nazwa                   & \textbf{Wyświatlenie użytkowników znajdujących się w domu}                                                                                                                                                                                                                                        \\
						\hline
						Opis                    & uzyskanie listy użytkowników obecnych w domu przez członka o odpowiednio nadanych uprawnieniach                                                                                                                                                                                                   \\
						\hline
						Aktorzy                 & \begin{itemize}\item Właściciel/współwłaściciel domu\end{itemize}                                                                                                                                                                                                                                 \\
						\hline
						Warunki początkowe      & Wirtualny dom posiada dodanych członków                                                                                                                                                                                                                                                           \\
						\hline
						Warunki końcowe sukcesu & Użytkownik otrzymuję listę osób znajdujących się aktualnie w domu i należących do wirtualnego domu                                                                                                                                                                                                \\
						\hline
						Warunki końcowe porażki & Użytkownik nie dostaje listy osób znajdujących się w domu                                                                                                                                                                                                                                         \\
						\hline
						Scenariusz główny       & \begin{enumerate}\item Użytkownik wybiera opcję "Aktywni domownicy"

\item Użytkownik wysyła zapytanie do serwera o wyświetlenie listy użytkowników znajdujących się w domu

\item Aplikacja wyświetla obecnych w domu użytkowników

\item Użytkownik zamyka wyświetlony komunikat\end{enumerate} \\
						\hline
						Scenariusz alternatywny & W domu nie są obecni żadni domownicy, w aplikacji wyświetla się mówiący o tym stosowny komunikat                                                                                                                                                                                                  \\
						\hline
					\end{tabular}
				\end{table}

			\item \textbf{Sprawdzenie historii aktywności urządzeń w domu} - uzyskanie
				historii działania urządzenia wraz z członakami, którzy go użytkowali

				\begin{table}[H]
					\centering
					\begin{tabular}{|c|p{7cm}|}
						\hline
						Nazwa                   & \textbf{Sprawdzenie historii aktywności urządzeń w domu}                                                                                                                                                                                                                                                                                                                \\
						\hline
						Opis                    & uzyskanie historii działania urządzenia wraz z członakami, którzy go użytkowali                                                                                                                                                                                                                                                                                         \\
						\hline
						Aktorzy                 & \begin{itemize}\item Właściciel/współwłaściciel domu\end{itemize}                                                                                                                                                                                                                                                                                                       \\
						\hline
						Warunki początkowe      & Urządzenie, którego historię chcemy sprawdzić zostało wcześniej dodane to wirtualnego domu                                                                                                                                                                                                                                                                              \\
						\hline
						Warunki końcowe sukcesu & Użytkownik otrzymuje historię działania urządzenia                                                                                                                                                                                                                                                                                                                      \\
						\hline
						Warunki końcowe porażki & Użytkownikowi nie zostaje wyświetlona historia urządzenia                                                                                                                                                                                                                                                                                                               \\
						\hline
						Scenariusz główny       & \begin{enumerate}\item Użytkownik wybiera opcję "Historia domu"

\item Użytkownik wybiera urządzenie którego historie chce zobaczyć

\item Użytkownik wysyła zapytanie do serwera o wyświetlenie statystyk aktywności wybranego urządzenia

\item Aplikacja wyświetla historię aktywności wybranego urządzenia

\item Użytkownik zamyka wyświetlone okno\end{enumerate} \\
						\hline
						Scenariusz alternatywny & -                                                                                                                                                                                                                                                                                                                                                                       \\
						\hline
					\end{tabular}
				\end{table}

			\item \textbf{Smart House Wrapped\textsuperscript{TM} - statystyki na
				koniec miesiąca} - uzyskanie statystyk na najróżniesze tematy o
				wszystkich członakach domu

				\begin{table}[H]
					\centering
					\begin{tabular}{|c|p{7cm}|}
						\hline
						Nazwa                   & \textbf{Smart House Wrapped\textsuperscript{TM} - statystyki na koniec miesiąca}                                                                                                                                                                                                                                                                                                                                                                                                                                                                                                                          \\
						\hline
						Opis                    & uzyskanie statystyk na najróżniesze tematy o wszystkich członakach domu                                                                                                                                                                                                                                                                                                                                                                                                                                                                                                                                   \\
						\hline
						Aktorzy                 & \begin{itemize}\item Właściciel/współwłaściciel domu\end{itemize}                                                                                                                                                                                                                                                                                                                                                                                                                                                                                                                                         \\
						\hline
						Warunki początkowe      & Wirtualny dom był już wcześniej wykorzystywany w życiu codziennym i zdążył zgromadzić statystyki                                                                                                                                                                                                                                                                                                                                                                                                                                                                                                          \\
						\hline
						Warunki końcowe sukcesu & Wyświetlono podsumowanie miesiąca                                                                                                                                                                                                                                                                                                                                                                                                                                                                                                                                                                         \\
						\hline
						Warunki końcowe porażki & Podsumowanie miesiąca nie zostało wyświetlone                                                                                                                                                                                                                                                                                                                                                                                                                                                                                                                                                             \\
						\hline
						Scenariusz główny       & \begin{enumerate}\item Funkcja pojawia się zawsze pod koniec miesiąca w menu aplikacji

\item Użytkownik wybiera opcję "Smart House Wrapped\textsuperscript{TM}"

\item Aplikacja wyświetla użytkownikowi prezentację złożoną z najważniejszych statystyk domu z minionego miesiąca, w tle prezentacji gra nastrojowa muzyka

\item Użytkownik ma możliwość przewijania prezentacji w przód lub w tył

\item Użytkownik przeszedł wszystkie slajdy, prezentacja kończy się

\item Użytkownik ma możliwość udostępnienia swojego "Smart House Wrapped\textsuperscript{TM}" w social mediach\end{enumerate} \\
						\hline
						Scenariusz alternatywny & \begin{itemize}\item AD. 4: Użytkownik nie wykona żadnej akcji, po jakimś czasie prezentacja sama przechodzi do następnego slajdu

\item AD. 6: Użytkownik nie chcę udostępnić swojego "Smart House Wrapped\textsuperscript{TM}", po zakończeniu prezentacji po prostu ją zamyka\end{itemize}                                                                                                                                                                                                                                                                                                             \\
						\hline
					\end{tabular}
				\end{table}

			\item \textbf{Wyświetlanie aktualnego stanu zabezpieczeń} - podgląd stanu
				zabezpieczeń (działanie kamer, opuszczenie rolet, włączenie świateł)

				\begin{table}[H]
					\centering
					\begin{tabular}{|c|p{7cm}|}
						\hline
						Nazwa                   & \textbf{Wyświetlanie aktualnego stanu zabezpieczeń}                                                                                                                                                                                                                                                                                                                                                                                                              \\
						\hline
						Opis                    & podgląd stanu zabezpieczeń (działanie kamer, opuszczenie rolet, włączenie świateł)                                                                                                                                                                                                                                                                                                                                                                               \\
						\hline
						Aktorzy                 & \begin{itemize}\item Właściciel/współwłaściciel domu\end{itemize}                                                                                                                                                                                                                                                                                                                                                                                                \\
						\hline
						Warunki początkowe      & Dom posiada dodane zabezpieczenia                                                                                                                                                                                                                                                                                                                                                                                                                                \\
						\hline
						Warunki końcowe sukcesu & Użytkownik uzyskuje informacje o stanie zabezpieczeń jego domu                                                                                                                                                                                                                                                                                                                                                                                                   \\
						\hline
						Warunki końcowe porażki & Użytkownikowi nie zostaje wyświetlona informacja o zabezpieczeniach                                                                                                                                                                                                                                                                                                                                                                                              \\
						\hline
						Scenariusz główny       & \begin{enumerate}\item Użytkownik wybiera opcje "Bezpieczeństwo"

\item Użytkownik wybiera opcje "Sprawdź aktualny stan zabezpieczeń"

\item Użytkownik wysyła zapytanie do serwera o stan zabezpieczeń

\item Serwer pobiera dane dotyczące stanu innych urządzeń domowych(kamer, rolet, świateł)

\item Aplikacja wyświetla stan wyżej wymienionych urządzeń domowych

\item Użytkownik zamyka okno stanu zabezpieczeń po zakończeniu oglądania\end{enumerate} \\
						\hline
						Scenariusz alternatywny & \begin{itemize}\item AD. 5: Aplikacja wyświetla stosowny komunikat, jeśli żadne z urządzeń bezpieczeństwa domowego nie zostało odnalezione\end{itemize}                                                                                                                                                                                                                                                                                                          \\
						\hline
					\end{tabular}
				\end{table}

			\item \textbf{Rejestracja asystenta głosowego innej firmy w naszym domu} -
				zainstalowanie, innego niż domyślny, asystenta głosowego do komunikacji
				z systemem

				\begin{table}[H]
					\centering
					\begin{tabular}{|c|p{7cm}|}
						\hline
						Nazwa                   & \textbf{Rejestracja asystenta głosowego innej firmy w naszym domu}                                                                                                                                                                                                                                                                                                                                 \\
						\hline
						Opis                    & zainstalowanie, innego niż domyślny, asystenta głosowego do komunikacji z systemem                                                                                                                                                                                                                                                                                                                 \\
						\hline
						Aktorzy                 & \begin{itemize}\item Właściciel/współwłaściciel domu\end{itemize}                                                                                                                                                                                                                                                                                                                                  \\
						\hline
						Warunki początkowe      & Asystent głosowy, który nas interesuje znajduje się w Marketplace                                                                                                                                                                                                                                                                                                                                  \\
						\hline
						Warunki końcowe sukcesu & Asystent głosowy zostaje zmieniony na ten wybrany przez użytkownika                                                                                                                                                                                                                                                                                                                                \\
						\hline
						Warunki końcowe porażki & Nie udało się zmienić asystenta głosowego                                                                                                                                                                                                                                                                                                                                                          \\
						\hline
						Scenariusz główny       & \begin{enumerate}\item Przejście przez użytkownika do Marketplace

\item Wyszukanie interesującego go asystenta głosowego

\item Pobranie wyszukanego przez użytkownika asystenta

\item Przejście przez użytkownika do ustawień aplikacji

\item Wybranie opcji "ustawienia dźwięku"

\item Zmiana asystenta głosowego przez użytkownika na tego pobranego wcześniej z Marketplace\end{enumerate} \\
						\hline
						Scenariusz alternatywny & -                                                                                                                                                                                                                                                                                                                                                                                                  \\
						\hline
					\end{tabular}
				\end{table}

			\item \textbf{Wysłanie zapytania lub polecenia w języku natrualnym do
				wybranego asystenta} - asystent interpretuje zapytanie i realizuje
				odpowiednie instrukcje w oparciu o inne use-casy

				\begin{table}[H]
					\centering
					\begin{tabular}{|c|p{7cm}|}
						\hline
						Nazwa                   & \textbf{Wysłanie zapytania lub polecenia w języku natrualnym do wybranego asystenta}                                                                                                                                                                                                                                                     \\
						\hline
						Opis                    & asystent interpretuje zapytanie i realizuje odpowiednie instrukcje w oparciu o inne use-casy                                                                                                                                                                                                                                             \\
						\hline
						Aktorzy                 & \begin{itemize}\item Właściciel/współwłaściciel domu\end{itemize}                                                                                                                                                                                                                                                                        \\
						\hline
						Warunki początkowe      & Wirtualny dom posiada zainstalowanego odpowiedniego asystenta głosowego                                                                                                                                                                                                                                                                  \\
						\hline
						Warunki końcowe sukcesu & Zapytanie użytkownika zostaje zrealizowane                                                                                                                                                                                                                                                                                               \\
						\hline
						Warunki końcowe porażki & Nie udało się zrealizować zapytania użytkownika                                                                                                                                                                                                                                                                                          \\
						\hline
						Scenariusz główny       & \begin{enumerate}\item Użytkownik wypowiada polecenie lub zapytanie zaczynając swoje zdanie od słowa "SmartHouse"

\item Asystent nasłuchuje komendy użytkownika i stara się dopasować ją do jednej z funkcjonalności systemu

\item Na podstawie pytania lub polecenia użytkownika asystent wykonuje oczekiwaną czynność\end{enumerate} \\
						\hline
						Scenariusz alternatywny & \begin{itemize}\item AD. 2: Asystent nie jest w stanie dopasować słów użytkownika do funkcjonalności systemu, na ekranie aplikacji pojawia się prośba o powtórzenie polecenia\end{itemize}                                                                                                                                               \\
						\hline
					\end{tabular}
				\end{table}

			\item \textbf{Przesyłanie danych do firm zewnętrznych z danymi do
				rozliczeń} - przesyłanie np. danych odczytanych z licznika prądu lub wody

				\begin{table}[H]
					\centering
					\begin{tabular}{|c|p{7cm}|}
						\hline
						Nazwa                   & \textbf{Przesyłanie danych do firm zewnętrznych z danymi do rozliczeń}                                                                                                                                                                                                                                                    \\
						\hline
						Opis                    & przesyłanie np. danych odczytanych z licznika prądu lub wody                                                                                                                                                                                                                                                              \\
						\hline
						Aktorzy                 & \begin{itemize}\item Właściciel/współwłaściciel domu\end{itemize}                                                                                                                                                                                                                                                         \\
						\hline
						Warunki początkowe      & Liczniki i urządzenia odczytu muszą być poprawnie zainstalowane i skonfigurowane                                                                                                                                                                                                                                          \\
						\hline
						Warunki końcowe sukcesu & Odczytane z liczników dane zostają odesłane do właściwych firm                                                                                                                                                                                                                                                            \\
						\hline
						Warunki końcowe porażki & Nie udaje się rozesłać odczytanych danych                                                                                                                                                                                                                                                                                 \\
						\hline
						Scenariusz główny       & \begin{enumerate}\item System odczytuje dane dotyczące zużycia zasobów przez dom z odpowiednich liczników

\item Dane zostają zapisane na serwerze aplikacji

\item Odczytane dane przesyłane są do zewnętrznych firm, czynność ta wykonywana jest automatycznie co określoną (w ustawieniach) ilość czasu\end{enumerate} \\
						\hline
						Scenariusz alternatywny & \begin{itemize}\item AD. 3: Adres firmy do której aplikacja ma dostarczyć dane jest nieprawidłowy, użytkownik otrzymuje stosowne powiadomienie, a dane nie zostają wysłane.\end{itemize}                                                                                                                                  \\
						\hline
					\end{tabular}
				\end{table}
		\end{enumerate}

		\subsubsection{Urządzenia, usprawnienia, marketplace}
		\begin{enumerate}
			\item \textbf{Wyświetlenie marketplace z dodatkowymi wtyczkami społeności}
				- pojawia się aktualna strona marketplace z odertami wtyczek opartymi o
				sugestie (dostępne urządzenia IoT oraz historię wyszukiwania)
				\begin{table}[H]
					\centering
					\begin{tabular}{|c|p{7cm}|}
						\hline
						Nazwa                   & \textbf{Wyświetlenie marketplace z dodatkowymi wtyczkami społeczności}                                                                                      \\
						\hline
						Opis                    & Pojawienie się aktualnej strony marketplace z ofertami wtyczek opartymi na sugestiach, uwzględniających dostępne urządzenia IoT oraz historię wyszukiwania. \\
						\hline
						Aktorzy                 & \begin{itemize}\item Członek wirtualnego domu\end{itemize}                                                                                                  \\
						\hline
						Warunki początkowe      & Użytkownik otwiera aplikację marketplace.                                                                                                                   \\
						\hline
						Warunki końcowe sukcesu & Użytkownik widzi stronę marketplace z wtyczkami społeczności.                                                                                               \\
						\hline
						Warunki końcowe porażki & Użytkownik nie widzi strony marketplace z wtyczkami społeczności.                                                                                           \\
						\hline
						Scenariusz główny       & \begin{enumerate}\item Użytkownik otwiera aplikację marketplace.

\item System wyświetla stronę marketplace z wtyczkami społeczności.\end{enumerate}        \\
						\hline
						Scenariusz alternatywny & \begin{enumerate}\item Użytkownik nie ma dostępu do internetu.

\item System wyświetla komunikat o braku połączenia z internetem.\end{enumerate}            \\
						\hline
					\end{tabular}
				\end{table}

			\item \textbf{Instalacja wtyczki} - wtyczka zostaje aktywowana, jej funkcje
				stają się dostępne
				\begin{table}[H]
					\centering
					\begin{tabular}{|c|p{7cm}|}
						\hline
						Nazwa                   & \textbf{Instalacja wtyczki}                                                                                                                                                                                                                                                                                                                   \\
						\hline
						Opis                    & Aktywacja wtyczki, co umożliwia dostęp do jej funkcji.                                                                                                                                                                                                                                                                                        \\
						\hline
						Aktorzy                 & \begin{itemize}\item Członek wirtualnego domu\end{itemize}                                                                                                                                                                                                                                                                                    \\
						\hline
						Warunki początkowe      & Użytkownik przegląda marketplace i wybiera wtyczkę.                                                                                                                                                                                                                                                                                           \\
						\hline
						Warunki końcowe sukcesu & Wtyczka została zainstalowana i udostępnia swoje funkcje.                                                                                                                                                                                                                                                                                     \\
						\hline
						Warunki końcowe porażki & Wtyczka nie została zainstalowana.                                                                                                                                                                                                                                                                                                            \\
						\hline
						Scenariusz główny       & \begin{enumerate}\item Użytkownik przegląda marketplace i wybiera wtyczkę.

\item Użytkownik wybiera opcję instalacji wtyczki.

\item System instaluje wtyczkę i udostępnia jej funkcje.\end{enumerate}                                                                                                                                       \\
						\hline
						Scenariusz alternatywny & \begin{enumerate}\item Następuje błąd instalacji.

\item System co jakiś czas automatycznie ponawia próby instalacji aż do skutku.

\item W przypadku bardzo wielu nieudanych prób oraz poprawnego połączenia z internetem, użytkownik zostaje powiadomiony o niedostępności wtyczki ze względów problemów po stronie serwera.\end{enumerate} \\
						\hline
					\end{tabular}
				\end{table}

			\item \textbf{Realizacja płatności za wybrane wtyczki użytkowników
				społeczności} - wykonanie płatności, w celu zdobycia możliwości używania
				wtyczki
				\begin{table}[H]
					\centering
					\begin{tabular}{|c|p{7cm}|}
						\hline
						Nazwa                   & \textbf{Realizacja płatności za wybrane wtyczki użytkowników społeczności}                                                                                                                        \\
						\hline
						Opis                    & Wykonanie płatności w celu uzyskania możliwości korzystania z wtyczki.                                                                                                                            \\
						\hline
						Aktorzy                 & \begin{itemize}\item Członek wirtualnego domu\end{itemize}                                                                                                                                        \\
						\hline
						Warunki początkowe      & Użytkownik wybiera wtyczkę do zakupu.                                                                                                                                                             \\
						\hline
						Warunki końcowe sukcesu & Wtyczka została aktywowana.                                                                                                                                                                       \\
						\hline
						Warunki końcowe porażki & Wtyczka nie została aktywowana.                                                                                                                                                                   \\
						\hline
						Scenariusz główny       & \begin{enumerate}\item Użytkownik wybiera wtyczkę do zakupu.

\item Użytkownik podaje dane płatnicze i akceptuje transakcję.

\item System przetwarza płatność i aktywuje wtyczkę.\end{enumerate} \\
						\hline
						Scenariusz alternatywny & \begin{enumerate}\item Następuje błąd płatności i transakcja zostaje anulowana.

\item Użytkownik zostanie powiadomiony o problemie z środkiem płatniczym.\end{enumerate}                         \\
						\hline
					\end{tabular}
				\end{table}

			\item \textbf{Dodanie własnej wtyczki na marketplace} - na marketplace
				pojawia się możliwość zakupu licencji użytkowania wtyczki
				\begin{table}[H]
					\centering
					\begin{tabular}{|c|p{7cm}|}
						\hline
						Nazwa                   & \textbf{Dodanie własnej wtyczki na marketplace}                                                                                                                                                                              \\
						\hline
						Opis                    & Pojawienie się możliwości zakupu licencji użytkowania nowej wtyczki na marketplace.                                                                                                                                          \\
						\hline
						Aktorzy                 & \begin{itemize}\item Twórca wtyczki\end{itemize}                                                                                                                                                                             \\
						\hline
						Warunki początkowe      & Twórca wtyczki dodaje ją do marketplace.                                                                                                                                                                                     \\
						\hline
						Warunki końcowe sukcesu & Użytkownicy mogą przeglądać i zakupić licencję na wtyczkę.                                                                                                                                                                   \\
						\hline
						Warunki końcowe porażki & Użytkownicy nie mogą przeglądać i zakupić licencji na wtyczkę.                                                                                                                                                               \\
						\hline
						Scenariusz główny       & \begin{enumerate}\item Twórca wtyczki dodaje ją do marketplace.

\item Użytkownicy mogą przeglądać i zakupić licencję na wtyczkę.\end{enumerate}                                                                             \\
						\hline
						Scenariusz alternatywny & \begin{enumerate}\item Opis wtyczki lub jej kod nie spełniają wymagań.

\item System odmawia twórcy wtyczki dodania jej do marketplace.

\item Użytkownicy nie mogą przeglądać i zakupić licencji na wtyczkę.\end{enumerate} \\
						\hline
					\end{tabular}
				\end{table}

			\item \textbf{Usunięcie wtyczki} - wtyczka zostaje zdezaktywowana, jej funkcje
				przestają być dostępne
				\begin{table}[H]
					\centering
					\begin{tabular}{|c|p{7cm}|}
						\hline
						Nazwa                   & \textbf{Usunięcie wtyczki}                                                                                                                                                                     \\
						\hline
						Opis                    & Dezaktywacja wtyczki, co powoduje zatrzymanie dostępu do jej funkcji.                                                                                                                          \\
						\hline
						Aktorzy                 & \begin{itemize}\item Członek wirtualnego domu\end{itemize}                                                                                                                                     \\
						\hline
						Warunki początkowe      & Użytkownik wybiera wtyczkę do usunięcia.                                                                                                                                                       \\
						\hline
						Warunki końcowe sukcesu & Wtyczka została zdezaktywowana.                                                                                                                                                                \\
						\hline
						Warunki końcowe porażki & Brak. Wtyczka zawsze zostanie usunięta.                                                                                                                                                        \\
						\hline
						Scenariusz główny       & \begin{enumerate}\item Użytkownik wybiera wtyczkę do usunięcia.

\item Użytkownik wybiera opcję usunięcia wtyczki.

\item System przerywa działanie wtyczki i ją odinstalowuje.\end{enumerate} \\
						\hline
						Scenariusz alternatywny & -                                                                                                                                                                                              \\
						\hline
					\end{tabular}
				\end{table}

			\item \textbf{Dodanie konta innego członka rodziny domu} - przypisaie
				członka do domu, uzyskuje podstawowe permisje, jak np. wstęp do domu
				\begin{table}[H]
					\centering
					\begin{tabular}{|c|p{7cm}|}
						\hline
						Nazwa                   & \textbf{Dodanie konta innego członka rodziny domu}                                                                                                                                                                                                                \\
						\hline
						Opis                    & Przypisanie członka do domu, nadając mu podstawowe uprawnienia, np. dostęp do domu.                                                                                                                                                                               \\
						\hline
						Aktorzy                 & \begin{itemize}\item Właściciel domu

\item Członek rodziny\end{itemize}                                                                                                                                                                                          \\
						\hline
						Warunki początkowe      & Właściciel domu otwiera panel administracyjny.                                                                                                                                                                                                                    \\
						\hline
						Warunki końcowe sukcesu & Nowy członek otrzymuje zaproszenie i akceptuje je.                                                                                                                                                                                                                \\
						\hline
						Warunki końcowe porażki & Nowy członek nie zostaje dodany do domu.                                                                                                                                                                                                                          \\
						\hline
						Scenariusz główny       & \begin{enumerate}\item Właściciel domu otwiera panel administracyjny.

\item Właściciel dodaje nowego członka rodziny.

\item Nowy członek otrzymuje zaproszenie i akceptuje je.

\item System przypisuje nowemu członkowi podstawowe uprawnienia.\end{enumerate} \\
						\hline
						Scenariusz alternatywny & \begin{enumerate}\item Nowy członek odrzuca zaproszenie.\end{enumerate}                                                                                                                                                                                           \\
						\hline
					\end{tabular}
				\end{table}

			\item \textbf{Przypisanie urządzeń członkom wirtualnego domu} -
				przypisanie uprawnień do danego członka domu przez właściciela
				\begin{table}[H]
					\centering
					\begin{tabular}{|c|p{7cm}|}
						\hline
						Nazwa                   & \textbf{Przypisanie urządzeń członkom wirtualnego domu}                                                                                                                                                                 \\
						\hline
						Opis                    & Przypisanie uprawnień do urządzeń konkretnemu członkowi domu przez właściciela.                                                                                                                                         \\
						\hline
						Aktorzy                 & \begin{itemize}\item Właściciel domu

\item Członek rodziny\end{itemize}                                                                                                                                                \\
						\hline
						Warunki początkowe      & Właściciel domu otwiera panel administracyjny.                                                                                                                                                                          \\
						\hline
						Warunki końcowe sukcesu & Właściciel wybiera członka domu i nadaje mu uprawnienia do urządzenia.                                                                                                                                                  \\
						\hline
						Warunki końcowe porażki & Właściciel nie nadaje uprawnień do urządzenia.                                                                                                                                                                          \\
						\hline
						Scenariusz główny       & \begin{enumerate}\item Właściciel domu otwiera panel administracyjny.

\item Właściciel wybiera urządzenie do przypisania.

\item Właściciel wybiera członka domu i nadaje mu uprawnienia do urządzenia.\end{enumerate} \\
						\hline
						Scenariusz alternatywny & \begin{enumerate}\item Właściciel przerywa edycję uprawnień.\end{enumerate}                                                                                                                                             \\
						\hline
					\end{tabular}
				\end{table}

			\item \textbf{Konfiguracja uprawnień dla członka rodziny} - nadanie lub
				odebranie uprawnień członkowi domu przez właściciela
				\begin{table}[H]
					\centering
					\begin{tabular}{|c|p{7cm}|}
						\hline
						Nazwa                   & \textbf{Konfiguracja uprawnień dla członka rodziny}                                                                                                                                               \\
						\hline
						Opis                    & Nadanie lub odebranie uprawnień dla konkretnego członka domu przez właściciela.                                                                                                                   \\
						\hline
						Aktorzy                 & \begin{itemize}\item Właściciel domu

\item Członek rodziny\end{itemize}                                                                                                                          \\
						\hline
						Warunki początkowe      & Właściciel domu otwiera panel administracyjny.                                                                                                                                                    \\
						\hline
						Warunki końcowe sukcesu & Właściciel modyfikuje uprawnienia przypisane do członka domu.                                                                                                                                     \\
						\hline
						Warunki końcowe porażki & Właściciel nie modyfikuje uprawnień przypisanych do członka domu.                                                                                                                                 \\
						\hline
						Scenariusz główny       & \begin{enumerate}\item Właściciel domu otwiera panel administracyjny.

\item Właściciel wybiera członka domu.

\item Właściciel modyfikuje uprawnienia przypisane do członka domu.\end{enumerate} \\
						\hline
						Scenariusz alternatywny & \begin{itemize}\item Właściciel przerywa edycję uprawnień.\end{itemize}                                                                                                                           \\
						\hline
					\end{tabular}
				\end{table}

			\item \textbf{Dodanie smart urządzeń do domu} - dodanie nowego urządzenia
				do domu, w celu konfiguracji i użytkowania
				\begin{table}[H]
					\centering
					\begin{tabular}{|c|p{7cm}|}
						\hline
						Nazwa                   & \textbf{Dodanie smart urządzeń do domu}                                                                                                                                                                                                                     \\
						\hline
						Opis                    & Dodanie nowego urządzenia do domu w celu konfiguracji i użytkowania.                                                                                                                                                                                        \\
						\hline
						Aktorzy                 & \begin{itemize}\item Członek wirtualnego domu\end{itemize}                                                                                                                                                                                                  \\
						\hline
						Warunki początkowe      & Użytkownik otwiera panel dodawania urządzeń.                                                                                                                                                                                                                \\
						\hline
						Warunki końcowe sukcesu & System dodaje urządzenie do listy zainstalowanych.                                                                                                                                                                                                          \\
						\hline
						Warunki końcowe porażki & System nie dodaje urządzenia do listy zainstalowanych.                                                                                                                                                                                                      \\
						\hline
						Scenariusz główny       & \begin{enumerate}\item Użytkownik otwiera panel dodawania urządzeń.

\item Użytkownik wybiera rodzaj urządzenia do dodania.

\item Użytkownik podłącza nowe urządzenie do systemu.

\item System dodaje urządzenie do listy zainstalowanych.\end{enumerate} \\
						\hline
						Scenariusz alternatywny & \begin{enumerate}\item System z różnych przyczyn nie potrafi skomunikować się z urządzeniem

\item System wyświetla komunikat o błędzie.\end{enumerate}                                                                                                     \\
						\hline
					\end{tabular}
				\end{table}
		\end{enumerate}

		\subsubsection{Zarządzanie i bezpieczeństwo}

		\begin{enumerate}
			\item \textbf{Zarządzanie trybem awaryjnym} - włączenie lub wyłączenie
				trybu, w którym wyłączane sa urządzenia i odcinane są media (gaz i prąd,
				nie licząc oświetlenia)

				\begin{table}[H]
					\centering
					\begin{tabular}{|c|p{7cm}|}
						\hline
						Nazwa                   & Zarządzanie trybem awaryjnym                                                                                                                                                                                                                                                 \\
						\hline
						Opis                    & Włączenie lub wyłączenie trybu, w którym wyłączane sa urządzenia i odcinane są media (gaz i prąd, nie licząc oświetlenia)                                                                                                                                                    \\
						\hline
						Aktorzy                 & \begin{itemize}\item Właściciel wirtualnego domu\end{itemize}                                                                                                                                                                                                                \\
						\hline
						Warunki początkowe      & Dom posiada urządzenia wspierające tryb awaryjny                                                                                                                                                                                                                             \\
						\hline
						Warunki końcowe sukcesu & Zmieniono status trybu awaryjnego                                                                                                                                                                                                                                            \\
						\hline
						Warunki końcowe porażki & Nie udało się zmienić status trybu awaryjnego                                                                                                                                                                                                                                \\
						\hline
						Scenariusz główny       & \begin{enumerate}\item System wykrywa niebezpieczną sytuacje (pożar, wyciek gazu, zalanie domu lub spięcie elektryczne).

\item System włącza tryb awaryjny.

\item Kiedy użytkownik stwierdzi, że problem został rozwiązany manualnie wyłącza tryb awaryjny.\end{enumerate} \\
						\hline
						Scenariusz alternatywny & \begin{itemize}\item Ad (i) Użytkownik rozpoznaje sam zagrożenie i manualnie wysyła zapytanie do serwera, aby włączył tryb awaryjny, następnie przejście do scenariusza głównego Ad (ii).\end{itemize}                                                                       \\
						\hline
					\end{tabular}
				\end{table}

			\item \textbf{Zarządzanie trybem oszczędzania energii (wyjazdem na wakacje)}
				- konfiguracja trybu oszczędzania energii (czasu nieużywania do wyłączenia
				urządzenia, automatycznego wyłączania świateł, jeśli zrobi się widno na zewnątrz)

				\begin{table}[H]
					\centering
					\begin{tabular}{|c|p{7cm}|}
						\hline
						Nazwa                   & Zarządzanie trybem oszczędzania energii (wyjazdem na wakacje)                                                                                                                                                                                                          \\
						\hline
						Opis                    & Konfiguracja trybu oszczędzania energii (czasu nieużywania do wyłączenia urządzenia, automatycznego wyłączania świateł, jeśli zrobi się widno na zewnątrz)                                                                                                             \\
						\hline
						Aktorzy                 & \begin{itemize}\item Właściciel wirtualnego domu\end{itemize}                                                                                                                                                                                                          \\
						\hline
						Warunki początkowe      & Dom posiada urządzenia wspierające tryb oszczędzania energii                                                                                                                                                                                                           \\
						\hline
						Warunki końcowe sukcesu & Zmieniono status trybu oszczędzania energii                                                                                                                                                                                                                            \\
						\hline
						Warunki końcowe porażki & Nie udało się zmienić statusu trybu oszczędzania energii                                                                                                                                                                                                               \\
						\hline
						Scenariusz główny       & \begin{enumerate}\item System wykrywa, że przez ostatnie 15 minut nikt nie znajdował się w wirtualnym domu.

\item System włącza tryb oszczędzania energii.

\item Kiedy domownik wróci do domu system automatycznie wyłącza tryb oszczędzania energii.\end{enumerate} \\
						\hline
						Scenariusz alternatywny & \begin{itemize}\item Ad (iii) Użytkownik sam wyłącza tryb oszczędzenia energii mimo to, że jest poza domem.\end{itemize}                                                                                                                                               \\
						\hline
					\end{tabular}
				\end{table}

			\item \textbf{Zarządzanie trybem SafePet\textsuperscript{TM} (utrzymanie zwierząt
				pod naszą nieobecność)} - konfiguracja szczegółów dotyczących utrzymania
				zwierzątk (podawanie karmy i wody, czyszczenie kuwety, kontrola temperatury
				pomieszczeń)

				\begin{table}[H]
					\centering
					\begin{tabular}{|c|p{7cm}|}
						\hline
						Nazwa                   & Zarządzanie trybem SafePet\textsuperscript{TM} (utrzymanie zwierząt pod naszą nieobecność)                                                                                                                                                                                                                                                        \\
						\hline
						Opis                    & Konfiguracja szczegółów dotyczących utrzymania zwierzątk (podawanie karmy i wody, czyszczenie kuwety, kontrola temperatury pomieszczeń)                                                                                                                                                                                                           \\
						\hline
						Aktorzy                 & \begin{itemize}\item Właściciel wirtualnego domu\end{itemize}                                                                                                                                                                                                                                                                                     \\
						\hline
						Warunki początkowe      & Dom posiada urządzenia wspierające tryb SafePet\textsuperscript{TM} oraz w domu znajdują się zwierzęta                                                                                                                                                                                                                                            \\
						\hline
						Warunki końcowe sukcesu & Zmieniono status trybu SafePet\textsuperscript{TM}                                                                                                                                                                                                                                                                                                \\
						\hline
						Warunki końcowe porażki & Nie udało się zmienić statusu trybu SafePet\textsuperscript{TM}                                                                                                                                                                                                                                                                                   \\
						\hline
						Scenariusz główny       & \begin{enumerate}\item System wykrywa, że przez ostatnie 15 minut nikt nie znajdował się w wirtualnym domu.

\item System włącza tryb SafePet\textsuperscript{TM}.

\item Kiedy domownik wróci do domu system automatycznie wyłącza tryb SafePet\textsuperscript{TM}.\end{enumerate}                                                              \\
						\hline
						Scenariusz alternatywny & \begin{itemize}\item Użytkownik wysyła zapytanie do serwera, aby serwer włączył tryb SafePet\textsuperscript{TM}.

\item Serwer włącza tryb SafePet\textsuperscript{TM}

\item Użytkownik wysyła zapytanie do serwera, aby serwer wyłączył tryb SafePet\textsuperscript{TM}.

\item Serwer wyłącza tryb SafePet\textsuperscript{TM}.\end{itemize} \\
						\hline
					\end{tabular}
				\end{table}

			\item \textbf{Włączenie trybu alarmowego} - wzywa służby, trwale zapisuje nagrania
				z kamer (nie będą usunięte automatycznie po 30 dniach)

				\begin{table}[H]
					\centering
					\begin{tabular}{|c|p{7cm}|}
						\hline
						Nazwa                   & Włączenie trybu alarmowego                                                                                                                                                                                                       \\
						\hline
						Aktorzy                 & \begin{itemize}\item Właściciel wirtualnego domu\end{itemize}                                                                                                                                                                    \\
						\hline
						Warunki początkowe      & W domu dochodzi do włamania                                                                                                                                                                                                      \\
						\hline
						Warunki końcowe sukcesu & Zmieniono status trybu alarmowego                                                                                                                                                                                                \\
						\hline
						Warunki końcowe porażki & Nie udało się zmienić statusu trybu alarmowego                                                                                                                                                                                   \\
						\hline
						Opis                    & Wzywa służby, trwale zapisuje nagrania z kamer (nie będą usunięte automatycznie po 30 dniach)                                                                                                                                    \\
						\hline
						Scenariusz główny       & \begin{enumerate}\item System wykrywa próbę włamania

\item System włącza tryb alarmowy.

\item Kiedy użytkownik stwierdzi, że nie ma zagrożenia, wysyła zapytanie do serwera, aby serwer wyłączył tryb alarmowy.\end{enumerate} \\
						\hline
						Scenariusz alternatywny & \begin{itemize}\item Ad (i) Użytkownik wysyła zapytanie, aby system włączył tryb alarmowy

\item Przejście do scenariusza głównego Ad (ii).\end{itemize}                                                                         \\
						\hline
					\end{tabular}
				\end{table}

				\begin{figure}[H]
					\centering
					\includegraphics[width=0.41\paperwidth]{
						Diagram czynności - Włączenie trybu alarmowego.png
					}
				\end{figure}

				\begin{table}[H]
					\centering
					\begin{tabular}{|c|p{7cm}|}
						\hline
						Nazwa                   & Utworzenie tymczasowego kodu dostępu dla gości                                                                                                                                                                                                                                      \\
						\hline
						Aktorzy                 & \begin{itemize}\item Właściciel wirtualnego domu\end{itemize}                                                                                                                                                                                                                       \\
						\hline
						Warunki początkowe      & Właściciel domu chce wpuścić do domu gościa pod swoją nieobecność                                                                                                                                                                                                                   \\
						\hline
						Warunki końcowe sukcesu & Udało się wygenerować kod gościa                                                                                                                                                                                                                                                    \\
						\hline
						Warunki końcowe porażki & Nie udało się wygenerować kodu gościa                                                                                                                                                                                                                                               \\
						\hline
						Opis                    & utworzenie kodu, który pozwalałby użytkownikowi zalogowanemu na dostęp do domu i urządzeń, w zakresie określonym przez właściciela domu                                                                                                                                             \\
						\hline
						Scenariusz główny       & \begin{enumerate}\item Właściciel domu wysyła zapytanie do serwera, aby serwer wygenerował kod.

\item System generuje kod.

\item Właściciel udostępnia kod gościowi.

\item Gość wpisuje kod w aplikacji.

\item Serwer przyznaje gościowi tymczasowe uprawnienia.\end{enumerate} \\
						\hline
						Scenariusz alternatywny & Brak                                                                                                                                                                                                                                                                                \\
						\hline
					\end{tabular}
				\end{table}

				\begin{table}[H]
					\centering
					\begin{tabular}{|c|p{7cm}|}
						\hline
						Nazwa                   & Aktywacja trybu 'poza domem' w celu zwiększenia zabezpieczeń                                                                                                                                                                                                                           \\
						\hline
						Opis                    & w tym trybie alarm informuje od razu odpowiednie służby, zaplanowane w harmonogramie wydarzenia nie mają miejsca (nie podnoszą się rolety, nie zaświecają się światła) oraz urządzenia przechodzą z trybu czuwania w stan wyłączenia.                                                  \\
						\hline
						Aktorzy                 & \begin{itemize}\item Właściciel wirtualnego domu\end{itemize}                                                                                                                                                                                                                          \\
						\hline
						Warunki początkowe      & W domu nikogo nie ma                                                                                                                                                                                                                                                                   \\
						\hline
						Warunki końcowe sukcesu & Zmieniono status trybu 'poza domem'                                                                                                                                                                                                                                                    \\
						\hline
						Warunki końcowe porażki & Nie udało się zmienić statusu 'poza domem'                                                                                                                                                                                                                                             \\
						\hline
						Scenariusz główny       & \begin{enumerate}\item Użytkownik wysyła zapytanie do serwera, aby serwer włączył tryb 'Poza domem'

\item Serwer włącza tryb 'poza domem'

\item Użytkownik wysyła zapytanie do serwera, aby serwer wyłączył tryb 'Poza domem'

\item Serwer wyłącza tryb 'poza domem'\end{enumerate} \\
						\hline
						Scenariusz alternatywny & Jeżeli tryb 'poza domem' jest włączony i ktoś wszedł do domu serwer automatycznie wyłącza tryb 'poza domem'.                                                                                                                                                                           \\
						\hline
					\end{tabular}
				\end{table}

				\begin{table}[H]
					\centering
					\begin{tabular}{|c|p{7cm}|}
						\hline
						Nazwa                   & Edycja stanu zabezpieczeń domu                                                                                                                                                                                                                                                                                                \\
						\hline
						Opis                    & Możliwość konfiguracji trybu domyslnego, trybu 'poza domem' oraz aktualnego stanu zabezpieczeń.                                                                                                                                                                                                                               \\
						\hline
						Aktorzy                 & \begin{itemize}\item Właściciel wirtualnego domu\end{itemize}                                                                                                                                                                                                                                                                 \\
						\hline
						Warunki początkowe      & Właściciel domu ma zamiar zmienić ustawienia                                                                                                                                                                                                                                                                                  \\
						\hline
						Warunki końcowe sukcesu & Wyświetlono i/lub zmieniono ustawienia.                                                                                                                                                                                                                                                                                       \\
						\hline
						Warunki końcowe porażki & Nie udało się wyświetlić lub zmienić ustawień.                                                                                                                                                                                                                                                                                \\
						\hline
						Scenariusz główny       & \begin{enumerate}\item Użytkownik wysyła zapytanie do serwera, aby wyświetlił obecne ustawienia.

\item Serwer wysyła użytkownikowi informacje o obecnych ustawieniach.

\item Użytkownik wysyła zapytanie do serwera, aby zmienić część ustawień.

\item Serwer zmienia wybrane przez użytkownika ustawienia.\end{enumerate} \\
						\hline
						Scenariusz alternatywny & Brak                                                                                                                                                                                                                                                                                                                          \\
						\hline
					\end{tabular}
				\end{table}
		\end{enumerate}
	\end{enumerate}

	\begin{figure}[H]
		\centering
		\includegraphics[width=0.48\paperwidth]{
			Diagram czynności - Edycja stanu zabezpieczeń domu.png
		}
	\end{figure}

	\newpage

	\section{Diagramy przypadków użycia}

	\subsection{Logowanie i podstawowe opcje}
	\begin{figure}[H]
		\centering
		\noindent
		\makebox[\textwidth]{\includegraphics[width=0.8\paperwidth]{
			Logowanie i podstawowe opcje.png
		}}
		\captionsetup{labelformat=empty}
		\caption{Logowanie i podstawowe opcje}
	\end{figure}

	\newpage
	\subsection{Harmonogram, kontakty, powiadomienia}
	\begin{figure}[H]
		\centering
		\noindent
		\makebox[\textwidth]{\includegraphics[width=0.8\paperwidth]{
			Harmonogram,
			kontakty,
			powiadomienia.png
		}}
		\captionsetup{labelformat=empty}
		\caption{Harmonogram, kontakty, powiadomienia}
	\end{figure}

	\newpage
	\subsection{Wykorzystanie urządzeń IoT}
	\begin{figure}[H]
		\centering
		\noindent
		\makebox[\textwidth]{\includegraphics[width=0.8\paperwidth]{
			Wykorzystanie urządzeń IoT.png
		}}
		\captionsetup{labelformat=empty}
		\caption{Wykorzystanie urządzeń IoT}
	\end{figure}

	\newpage
	\subsection{Automatyzacja i statystyki}
	\begin{figure}[H]
		\centering
		\noindent
		\makebox[\textwidth]{\includegraphics[width=0.8\paperwidth]{
			Automatyzacja i statystyki.png
		}}
		\captionsetup{labelformat=empty}
		\caption{Automatyzacja i statystyki}
	\end{figure}

	\newpage
	\subsection{Urządzenia, uprawnienia, marketplace}
	\begin{figure}[H]
		\centering
		\noindent
		\makebox[\textwidth]{\includegraphics[width=0.8\paperwidth]{
			Urządzenia,
			uprawnienia,
			marketplace.png
		}}
		\captionsetup{labelformat=empty}
		\caption{Urządzenia, uprawnienia, marketplace}
	\end{figure}

	\newpage
	\subsection{Zarządzanie i bezpieczeństwo}
	\begin{figure}[H]
		\centering
		\noindent
		\makebox[\textwidth]{\includegraphics[width=0.8\paperwidth]{
			Zarządzanie i bezpieczeństwo.png
		}}
		\captionsetup{labelformat=empty}
		\caption{Zarządzanie i bezpieczeństwo}
	\end{figure}

	\newpage

	\section{Diagramy Sekwencji}

	\begin{figure}[H]
		\centering
		\noindent
		\makebox[\textwidth]{\includegraphics[width=0.8\paperwidth]{Kontakty.png}}
		\captionsetup{labelformat=empty}
		\caption{Konfiguracja listy kontaków do powiadamiania w razie nagłych
		sytuacji}
	\end{figure}

	\begin{figure}[H]
		\centering
		\noindent
		\makebox[\textwidth]{\includegraphics[width=0.8\paperwidth]{
			Powiadomienia.png
		}}
		\captionsetup{labelformat=empty}
		\caption{Wyciszanie powiadomień na określony czas}
	\end{figure}

	\newpage
	\section{Podział pracy}

	\begin{itemize}
		\item Wprowadzenie: \\ Rafał Żelazko \\ Artur Wojciechowski \\ Bartosz
			Sendor \\ Patryk Skowron \\ Bartosz Warchoł \\

		\item Aktorzy:\\ Rafał Żelazko \\ Bartosz Sendor \\ Patryk Skowron \\ Artur Wojciechowski
			\\

		\item Przypadki użycia:\\ Artur Wojciechowski \\ Bartosz Sendor \\ Patryk Skowron
			\\ Rafał Żelazko \\

		\item Diagramy przypadków użycia: Bartosz Warchoł

		\item Scenariusze przypadków użycia

			\begin{enumerate}
				\item Logowanie i podstawowe opcje: Rafał Żelazko

				\item Harmonogram, kontakty, powiadomienia: Bartosz Warchoł

				\item Wykorzystanie urzadzeń IoT: Bartosz Sendor

				\item Automatyzacja i statystyki: Artur Wojciechowski

				\item Urządzenia, usprawnienia, marketplace: Rafał Żelazko

				\item Zarządzanie i bezpieczeństwo: Patryk Skowron
			\end{enumerate}
	\end{itemize}
\end{document}